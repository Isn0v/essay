\subsection{Снижение места под данные трэйса}

Простое копирование необходимых бинарников, исполняемых файлов и библиотек быстро раздует память, 
поэтому вместо этого создаются жесткие ссылки 
на все необходимые файлы.

Также, современные файловые системы (\href{https://ru.wikipedia.org/wiki/XFS}{\texttt{XFS}})
поддерживают механизм \href{https://en.wikipedia.org/wiki/Copy-on-write}{\texttt{COW}} (copy-on-write). 
RR просто клонирует нужные области в каталоге трэйса (записанного наблюдаемого потока), не тратя почти места и времени. 
Фактическое копирование произойдёт лишь в случае модификации исходного файла.

\subsection{Эмуляция системных вызовов}

Практика показала, что простое использование \href{https://man7.org/linux/man-pages/man2/ptrace.2.html}{\texttt{ptrace}} 
неэффективно из-за частой смены контекста.
Инженерами \cite{rr-paper} были придуманы решения, которые позволяют эмулировать системные вызовы разными способами:

\begin{itemize}
  \item Ставится точка останова (breakpoint) на момент входа в некоторый системный вызов. 
  Программный счетчик наблюдаемого потока доходит до инструкции системного вызова и его исполнение останавливается. 
  RR это понимает, и в этот момент подменяет данные на те, что должны получиться в результате 
  исполнения этого системного вызова.
  \item В момент запуска RR размещает в адресном пространстве каждого наблюдаемого процесса специальную библиотеку 
  (interception library). Об этом в следующей секции.
\end{itemize}

\subsection{Interception Library}

Interception library — ключевая оптимизация в RR, позволяющая достичь низких накладных расходов за счёт минимизации взаимодействия с ptrace.
Это библиотека, внедряемая в записываемый процесс для перехвата системных вызовов непосредственно внутри него. 
В результате, с её помощью удается снизить количество смен контекста до с 2 до 0.
