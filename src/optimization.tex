\subsection{Снижение места под данные трэйса}

Простое копирование необходимых бинарников, исполняемых файлов и библиотек быстро раздует память, 
поэтому вместо этого создаются жесткие ссылки 
на все необходимые файлы.

Также, современные файловые системы (\texttt{XFS}\footnote{\url{https://ru.wikipedia.org/wiki/XFS}})
поддерживают механизм \texttt{COW}\footnote{\url{https://en.wikipedia.org/wiki/Copy-on-write}} (copy-on-write). 
RR просто клонирует нужные области в каталоге трэйса (записанного наблюдаемого потока), не тратя почти места и времени. 
Фактическое копирование произойдёт лишь в случае модификации исходного файла.

\subsection{Эмуляция системных вызовов}

Практика показала, что простое использование \texttt{ptrace} 
неэффективно из-за частой смены контекста.
Инженерами \cite{rr-paper} были придуманы решения, которые позволяют эмулировать системные вызовы разными способами:

\begin{itemize}
  \item Ставится точка останова (breakpoint) на момент входа в некоторый системный вызов. 
  Программный счетчик наблюдаемого потока доходит до инструкции системного вызова и его исполнение останавливается. 
  RR это понимает, и в этот момент подменяет данные на те, что должны получиться в результате 
  исполнения этого системного вызова.
  \item В момент запуска RR размещает в адресном пространстве каждого наблюдаемого процесса специальную библиотеку 
  (interception library). Об этом в следующей секции.
\end{itemize}

\subsection{Interception Library}

Interception library — ключевая оптимизация в RR, позволяющая достичь низких накладных расходов за счёт минимизации взаимодействия с 
\texttt{ptrace}.
Она внедряется в каждый записываемый процесс, выполняет системные вызова, не вызывая
\texttt{ptrace},
и записывает результаты исполнения в специальный буфер, разделяемый с RR.
В результате, с её помощью удается снизить количество смен контекста до с 2 до 0.

\subsection{Оптимизация чтения с клонированием блоков}

Если и трэйс, и входные данные находятся в одной файловой системе, поддерживающей copy-on-write, то в коде interception library 
системного вызова \texttt{read} реализуется клонирование блоков в трэйс файл, свой для каждого наблюдаемого потока.

