Запись и воспроизведение программы — процесс записи исполнения программы так, что 
она  может быть воспроизведена с помощью этого же инструмента 
с целью анализа и выявления ошибок.
Он является довольно удобным инструментом для тестирования и отладки. Тем не менее, 
хочется иметь возможность в не просто реализовать функции записи и воспроизведения, 
но и поддержать действительно мощные возможности отладки, 
такие как reverse debugging и time travel debugging.

Давайте рассмотрим, как к этой задаче подошли инженеры проекта Record-and-Replay
(Далее, RR) в своей статье \cite{rr-paper} о разработке этого инструмента.



% На данный момент существует много подходов для его реализации, 
% но все они имеют недостатки: где-то требуется 
% сохранять состояние всей виртуальной машины, где-то нужно модифицировать 
% ядро ОС или же компиляторы, в результате снижая область 
% применения или же увеличивая накладные расходы. Важно понять, как это сделать с 
% малым снижением или вовсе без потерь в производительности 
% с учетом того, чтобы отлаживаемая программа работала абсолютно идентично той, 
% что была записана на определенном устройстве, на всех поддерживаемых 
% устройствах.

% Здесь будет рассматриваться подход к разработке системы RR записи и повторного 
% воспроизведения программы без изменения пользовательского 
% пространства приложения, со стандартным набором инструментов Linux и базируясь на 
% x86/x86\_64 процессорах.