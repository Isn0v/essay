% Запись и воспроизведение программы — процесс записи исполнения программы так, что 
% она  может быть воспроизведена с помощью этого же инструмента 
% с целью анализа и выявления ошибок.
% Он является довольно удобным инструментом для тестирования и отладки. Тем не менее, 
% хочется иметь возможность не просто реализовать функции записи и воспроизведения, 
% но и поддержать действительно мощные возможности отладки, 
% такие как reverse debugging и time travel debugging.

% Давайте рассмотрим, как к этой задаче подошли инженеры проекта Record-and-Replay
% (Далее, RR) в своей статье \cite{rr-paper} о разработке этого инструмента.


Отладка программного обеспечения является неотъемлемой и зачастую наиболее
трудоемкой частью процесса разработки. На заре появления такой возможности,
традиционные отладчики, такие как
GDB\footnote{\url{https://sourceware.org/gdb/}} или
LLDB\footnote{\url{https://lldb.llvm.org/}}, имели фундаментальное ограничение:
возможность двигаться по коду преимущественно в одном направлении –
вперед\footnote{Поправка: на текущий момент (2025 год) и GDB, и LLDB могут
двигаться и вперед, и назад, но мы будем говорить о том, что они умели в
прошлом.}. Это существенно затрудняет поиск причин сложных, особенно
недерминированных или редко воспроизводимых ошибок, когда точные условия
возникновения сбоя трудно или невозможно воссоздать вручную.

Так появилась концепция Записи и Воспроизведения. Суть её заключается в
фиксации всех недетерминированных входных данных и событий (например, системных
вызовов, сигналов, планировщика потоков) во время выполнения программы таким
образом, чтобы это выполнение можно было впоследствии точно воспроизвести.
Инструмент, реализующий Записи и Воспроизведение, позволяет многократно
''проигрывать'' проблемный запуск программы, что является важным аспектом для
тестирования и выявления ошибок, гарантируя их стопроцентную повторяемость в
рамках записанного сеанса.

Тем не менее, возможности простого воспроизведения не всегда достаточны для
глубокого понимания причин сбоя. Современные требования к отладке диктуют
необходимость в более мощных инструментах. Разработчики стремятся не просто
повторить ошибку, а получить возможность ''путешествовать во времени'' по истории
выполнения программы: выполнять шаги назад (reverse debugging), устанавливать
точки останова на прошлые события и исследовать состояние программы в любой
произвольный момент времени (time travel debugging). Эти возможности позволяют
быстро локализовать источник ошибки, отслеживая, как некорректное состояние
программы возникло и распространялось.

Одним из наиболее ярких и эффективных инструментов, реализующих эти продвинутые
возможности на базе механизма записи и воспроизведения, является отладчик rr
(Record and Replay debugger), изначально разработанный в Mozilla. В данном
реферате мы подробно рассмотрим подходы и технические решения, предложенные
инженерами проекта rr (далее – RR) в их статье \cite{rr-paper}
для создания этого отладочного инструмента.