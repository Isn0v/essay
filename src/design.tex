\subsection{Record-and-Replay: способ записи и воспроизведения}

Задача Record — 
сохранить некую часть информации об исполнении программы и ее используемых данных, 
необходимую для того, чтобы в дальнейшем 
точно воспроизвести поведение записываемой программы.

Задача Replay — 
воспроизвести исполнение программы, сохраненное в ходе процесса записи, 
чтобы получить результаты, которые были получены во время записи.

Система RR ставит выполнение данных задач для программы 
с минимальными накладными расходами, без внесения каких-либо низкоуровневых изменений в 
окружение и используя лишь нативные возможности, предоставляемые операционной системой.

\subsection{Ограничения исполнения}

В системе RR используется модель записи исполнения потоков N:1. Это означает, 
что из всех N имеющихся потоков, одновременно исполняться и, соответственно, записываться, будет лишь один поток.
Причина такого решения в сложности 
контроля исполнения многопоточных программ. Во время записи из-за многопоточности некоторые данные могут быть либо записаны некорректно, 
либо вовсе потеряны, что недопустимо. Во время воспроизведения требуется корректная синхронизация потоков, 
добиться которой может быть сложно.

% Несмотря на решение большинства проблем с многопоточностью, этот подход приводит к ограничениям в определении 
% некоторых многопоточных ошибок, таких как \textit{deadlock}, \textit{starvation}, \textit{liveness} и т.п.

\subsection{Детерминизм}

При воспроизведении любой произвольной записанной программы необходимо воспроизвести ее в точном совпадении. 
Это означает, что требуется линеаризовать или же точно определять ход исполнения программы в каждый момент времени.
Это не такая простая задача, как может показаться на первый взгляд, 
так как детерминизм должен поддерживаться за счет контроля со стороны пользовательской программы, 
не имеющей доступа к большинству возможностей ядра ОС.
Конкретный способ реализации будет рассмотрен далее.
